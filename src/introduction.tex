
\section*{Introduction}
\label{sec:intro}

Le but de ce projet est de résoudre, à l'aide d'un SMA, un problème
d'allocation de ressources si elles ne sont pas partageable.

Les contraintes importantes que nous avons retenues et qui ont
motivées les choix que nous avons fait dans ce projet sont les
suivantes:

\begin{itemize}
\item Trouver une place le plus rapidement possible.
\item Envoyer le moins de messages possibles.
\item Conserver un protocole suffisamment simple pour être le plus
  rapide possible.
\item Se baser sur le comportement humain (plus cohérent que des
  fourmis pour une recherche de place).
\end{itemize}

Nous avons donc tenté d'avoir un comportement permettant de respecter
toutes ces contraintes.

Dans une première partie nous présenterons notre solution, dans une
seconde partie nous rentrerons plus en détail dans l'implémentation,
et dans une troisième partie, nous présenterons notre analyse de
résultat à l'aide de benchmark. Pour évaluer la validité de notre
solution, nous avons évalué le temps moyen par agent pour trouver une
place, ainsi que le ratio ``temps à chercher / (temps à chercher +
temps garé)'', et enfin le nombre de message.

%%% Local Variables:
%%% mode: latex
%%% TeX-master: "../sma"
%%% compile-command: "cd ..; make"
%%% End:
